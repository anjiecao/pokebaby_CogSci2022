% Template for Cogsci submission with R Markdown

% Stuff changed from original Markdown PLOS Template
\documentclass[10pt, letterpaper]{article}

\usepackage{cogsci}
\usepackage{pslatex}
\usepackage{float}
\usepackage{caption}

% amsmath package, useful for mathematical formulas
\usepackage{amsmath}

% amssymb package, useful for mathematical symbols
\usepackage{amssymb}

% hyperref package, useful for hyperlinks
\usepackage{hyperref}

% graphicx package, useful for including eps and pdf graphics
% include graphics with the command \includegraphics
\usepackage{graphicx}

% Sweave(-like)
\usepackage{fancyvrb}
\DefineVerbatimEnvironment{Sinput}{Verbatim}{fontshape=sl}
\DefineVerbatimEnvironment{Soutput}{Verbatim}{}
\DefineVerbatimEnvironment{Scode}{Verbatim}{fontshape=sl}
\newenvironment{Schunk}{}{}
\DefineVerbatimEnvironment{Code}{Verbatim}{}
\DefineVerbatimEnvironment{CodeInput}{Verbatim}{fontshape=sl}
\DefineVerbatimEnvironment{CodeOutput}{Verbatim}{}
\newenvironment{CodeChunk}{}{}

% cite package, to clean up citations in the main text. Do not remove.
\usepackage{apacite}

% KM added 1/4/18 to allow control of blind submission


\usepackage{color}

% Use doublespacing - comment out for single spacing
%\usepackage{setspace}
%\doublespacing


% % Text layout
% \topmargin 0.0cm
% \oddsidemargin 0.5cm
% \evensidemargin 0.5cm
% \textwidth 16cm
% \textheight 21cm

\title{How to Make a Proceedings Paper Submission}


\author{{\large \bf Morton Ann Gernsbacher (MAG@Macc.Wisc.Edu)} \\ Department of Psychology, 1202 W. Johnson Street \\ Madison, WI 53706 USA \AND {\large \bf Sharon J.~Derry (SDJ@Macc.Wisc.Edu)} \\ Department of Educational Psychology, 1025 W. Johnson Street \\ Madison, WI 53706 USA}


\begin{document}

\maketitle

\begin{abstract}
Include no author information in the initial submission, to facilitate
blind review. The abstract should be one paragraph, indented 1/8 inch on
both sides, in 9\textasciitilde point font with single spacing. The
heading `Abstract' should be 10\textasciitilde point, bold, centered,
with one line of space below it. This one-paragraph abstract section is
required only for standard six page proceedings papers. Following the
abstract should be a blank line, followed by the header `Keywords' and a
list of descriptive keywords separated by semicolons, all in
9\textasciitilde point font, as shown below.

\textbf{Keywords:}
Add your choice of indexing terms or keywords; kindly use a semi-colon;
between each term.
\end{abstract}

\hypertarget{introduction}{%
\section{Introduction}\label{introduction}}

\hypertarget{experiment}{%
\section{Experiment}\label{experiment}}

\hypertarget{methods}{%
\subsection{Methods}\label{methods}}

\hypertarget{participants}{%
\subsubsection{Participants}\label{participants}}

We recruited 450 participants (Age range: M = ;) on Prolific. They were
randomly assigned to one of the three conditions of the experiment
(Curiosity: N =; Memory: N = ; Math: N =; ). Participants were excluded
if they showed irregular reaction times (N = ???) or their responses in
the filler tasks indicates low engagement with the experiment
(Curiosity: N =; Memory: N = ; Math: N =; ). All exclusion criteria were
pre-registered. The final sample included N participants (Curiosity N =
; Memory: N =; Math: N =).

\hypertarget{procedure}{%
\subsubsection{Procedure}\label{procedure}}

This is a web-based self-paced visual presentation task. Participants
were instructed to look at a sequence of animated creatures at their own
pace and answer some questions throughout. At the end of the experiment,
participants were asked to rate the similarity between pairs of
creatures and complexity of creatures they encountered on a 7-point
Likert Scale. Each participant saw eight blocks in total, half of which
used creatures with high perceptual complexity, and half of which used
creatures with low perceptual complexity. On each trial, an animated
creature showed up on the screen. participants can press the down arrow
to go to the next trial whenever they want after a minimum viewing time
of 500 ms.

Each block consisted of six trials. A trial can be either a background
trial (B) or a deviant trial (D). A background trial presented a
creature repeatedly, and the deviant trial presented a different
creature from the background trial in the block. Two creatures in the
blocks were matched for visual complexity. There were four sequences of
background trials and deviant trials. Each sequence appeared twice, once
with high complexity stimuli and once with low complexity stimuli. The
deviant trial can appear at either the second (BDBBBB), the fourth
(BBBDBB), or the sixth trial (BBBBBD) in the block. Two blocks do not
have deviant trials (BBBBBB). The creatures presented in the deviant
trials and background trials were matched for complexity.

Participants were randomly assigned to one of the three conditions:
Curiosity, Memory, and Math The three conditions only differed in the
type of questions asked following each block. In Curiosity condition,
participants were asked to rate ``How curious are you about the
creature?'' on a 5-point Likert scale. In Memory condition, a
forced-choice recognition question followed each block (``Have you seen
this creature before?''). The creature used in the question in both
conditions was either a creature presented in the preceding block or a
novel creature matched in complexity. In Math condition, the
participants were asked a simple arithmetic question (``What is 5 +
7?'') in multiple-choice format.

\hypertarget{stimuli}{%
\subsubsection{Stimuli}\label{stimuli}}

The animated creatures (Fig 1) were created using Spore (a game
developed by Maxis in 2008). There were forty creatures in total, half
of which have low perceptual complexity (e.g.~the creatures do not have
limbs, additional body parts, facial features, or textured skin), and
half of which have high perceptual complexity (i.e.~they do have the
aforementioned features). We used the ``animated avatar'' function in
Spore to capture the creatures in motion.

\hypertarget{results}{%
\subsection{Results}\label{results}}

\hypertarget{analytic-approach}{%
\subsubsection{Analytic Approach}\label{analytic-approach}}

The sample size, methods, and main analyses were all pre-registered and
are available at {[}LINK{]}. Data and analysis scripts are available at
{[}LINK{]}.

\hypertarget{manipulation-check}{%
\subsubsection{Manipulation Check}\label{manipulation-check}}

The complex animated creatures were rated as more perceptually complex
(M = ; SD = ) than the simple animated creatures (M = ; SD = ). Pairs of
background creature and deviant creature were rated as moderately
dissimilar to one another (M = ; SD = ).

\hypertarget{evaluating-the-paradigm}{%
\subsubsection{Evaluating the Paradigm}\label{evaluating-the-paradigm}}

Three criteria were selected to evaluate whether the paradigms
successfully captured the characteristic looking time patterns observed
in infant literature: habituation (the decrease in looking time for a
stimulus with repeated presentations), dishabituation (the increase in
looking time to a new stimulus after habituated to one stimulus), and
complexity effect (longer looking time for perceptually more complex
stimuli). To evaluate the phenomenon quantitatively, we ran a linear
mixed effects model with maximal random effect structure. {[}DESCRIBE
THE MODEL{]}. {[}REPORT THE MODEL RESULTS{]}

\hypertarget{order-effect}{%
\subsubsection{Order Effect}\label{order-effect}}

While visualizing the data, we unexpectedly found that the position in
which the deviant trial appeared in the sequence had an effect on the
shape of the habituation and dishabituation curves. To explore this
phenomenon quantitatively, we operationalized the magnitude of
dishabituation as the difference between the looking time at the deviant
trial minus the background trial at the same position. Then, we fit a
mixed effect model with the position of deviant as fixed effect and
{[}???{]} as a random effect. We found that the position was a
significant predictor of the magnitude of dishabituation (looking time
at the deviant trial minus the background trial at the same position).
Deviant trials that appeared last elicited the strongest dishabituation
effect (M = ; SD:, ), followed by the deviant trials appeared fourth (M,
SD), with the deviant trial on the second showing the smallest amount of
dishabituation (M, SD).

\hypertarget{discussion}{%
\subsection{Discussion}\label{discussion}}

\hypertarget{model}{%
\section{Model}\label{model}}

\hypertarget{general-discussion}{%
\section{General Discussion}\label{general-discussion}}

\hypertarget{references}{%
\section{References}\label{references}}

\hypertarget{references-1}{%
\section{References}\label{references-1}}

\setlength{\parindent}{-0.1in} 
\setlength{\leftskip}{0.125in}

\noindent

\bibliographystyle{apacite}


\end{document}
